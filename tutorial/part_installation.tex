% SPDX-License-Identifier: MIT
% Copyright (c) 2017-2020 Forschungszentrum Juelich GmbH
% This code is licensed under MIT license (see the LICENSE file for details)
%
\part{Installation}
\makepart

\section{File Content}
\begin{frame}[fragile]
        \frametitle{beamertheme-juelich.zip}
        The .zip archive consists of 1 directory with 2 subdirectories.
        \begin{itemize}
      \item \verb+beamertheme-juelich.zip+
      \item \verb+beamertheme-juelich/+ \hfill main directory of the .zip file
      \begin{itemize}
        \item \verb+/+ \hfill directory containing the .sty files
        \begin{itemize}
          \item \verb+beamerthemeJuelich.sty+ \hfill main style file
          \item \verb+beamercolorthemeJuelich.sty+ \hfill aux. style file
          \item \verb+beamerfontthemeJuelich.sty+ \hfill aux. style file
          \item \verb+beamerinnerthemeJuelich.sty+ \hfill aux. style file
          \item \verb+beamerouterthemeJuelich.sty+ \hfill aux. style file
          \item \verb+fzj.pdf+ \hfill Juelich logo for pdf\LaTeX
        \end{itemize}
        \item[]
        \item \verb+tutorial/+ \hfill directory containing some minimal examples and the sources of this tutorial
        \item[]
        \item \verb+tests/+ \hfill directory containing test infrastructure for the theme
      \end{itemize}
    \end{itemize}
\end{frame}

\section{Installation}
\subsection{On Linux-based Machines}
\begin{frame}[fragile]
        \frametitle{Linux Installation}
        \framesubtitle{Choose {\tt texmf} Tree}
        First, choose your favorite install directory.  \newline
        Then, create a new subdirectory \verb+beamertheme-juelich+
        \begin{block}{Change to your {\tt texmf} tree and create subdirectory}
    \begin{itemize}
      \item \verb+cd $HOME/texmf/tex/latex/+ \hfill \textcolor{fzjblue}{[preferred]} or
      \item \verb+cd /usr/share/texmf/tex/latex/+ \hfill or
      \item \verb+cd /usr/local/share/texmf/tex/latex/+ \hfill
      \item[]
      \item \verb+mkdir beamertheme-juelich+
    \end{itemize}
    \end{block}
\end{frame}

\begin{frame}[fragile]
        \frametitle{Linux: Install the {\tt .sty} Files}
        \begin{block}{Create Directory + Copy files + Update \TeX}
        \begin{itemize}
                \item Unzip \verb+beamertheme-juelich.zip+ file
                \item Copy all files from subdirectory \verb+beamertheme-juelich/+ into
                the new subdirectory \verb+beamertheme-juelich+
                \item[]
                \item Try to compile the minimal examples in the \verb+tutorial/+ subdirectory
                \item[] \verb+pdflatex minimal.tex+
                \item[] \verb+pdflatex minimal_handout.tex+
                \item Afterwards try to compile this tutorial in the \verb+tutorial/+ subdirectory
                \item[] \verb+pdflatex tutorial.tex+
        \end{itemize}
    \end{block}
    \begin{alertblock}{Supported flavors of \LaTeX{}}
    Pure \verb+latex+ is not supported, please use either \verb+pdflatex+, \verb+xelatex+ or \verb+lualatex+
    \end{alertblock}
\end{frame}

\section{Test Installation I}
\begin{frame}[fragile,label=examples]
        \frametitle{Test your Installation}
        \framesubtitle{Try to compile this minimal talk: {\tt tutorial/minimal.tex}}
        \begin{columns}
        \begin{column}[T]{0.4\textwidth}
        \footnotesize
        \verbatiminput{./minimal}
        \end{column}\hfill
        \begin{column}[T]{0.4\textwidth}
        \setlength{\fboxsep}{0pt}%
        \fbox{\includegraphics[width=\textwidth]{./minimal}}
        \end{column}
        \end{columns}
\end{frame}

\begin{frame}[fragile]
        \frametitle{Test your Installation II}
        \framesubtitle{Try to compile this minimal talk with handouts: {\tt tutorial/minimal\_handout.tex}}
        \begin{columns}
        \begin{column}[T]{0.4\textwidth}
        \scriptsize
        \verbatiminput{./minimal_handout}
        \end{column}
        \begin{column}[T]{0.4\textwidth}
        \includegraphics[height=0.7\textheight]{./minimal_handout}
        \end{column}
        \end{columns}
\end{frame}

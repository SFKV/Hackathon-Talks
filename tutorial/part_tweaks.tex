% !TEX root = tutorial.tex
% SPDX-License-Identifier: MIT
% Copyright (c) 2017-2020 Forschungszentrum Juelich GmbH
% This code is licensed under MIT license (see the LICENSE file for details)
%
\part{Tweaks}
\makepart
\section{Slide Number Display}

\begin{frame}[fragile,label=tweaks]
        \frametitle{Slide Number Display}
        \framesubtitle{How to change the slide number style}
        \begin{block}{Full Display: Current Slide | Overall Number of Slides}
                \scriptsize\verb+\setbeamertemplate{frame number}[full]+ \hfill
                \scriptsize\usebeamercolor[fg]{frametitle} Slide 42 $|$ 524
        \end{block}
        \begin{block}{No Display: empty}
                \scriptsize\verb+\setbeamertemplate{frame number}[empty]+ \hfill
                \scriptsize\usebeamercolor[fg]{frametitle}
        \end{block}
        \begin{block}{Default Display: Current Slide}
                \scriptsize\verb+\setbeamertemplate{frame number}[default]+ \hfill
                \scriptsize\usebeamercolor[fg]{frametitle} Slide 42
        \end{block}
        \begin{block}{Translation}
        If you choose german as language the name \emph{Slide} will be translated
        to \emph{Folie} automatically (See \hyperlink{translation}{\alert{this}}
        slide)
        \end{block}
\end{frame}


\section{Partner Logos}

\setbeamertemplate{footer element1}[logo]{jara}%
\setbeamertemplate{footer element3}[logo]{uni_bonn}%
\setbeamertemplate{footer element2}[logo]{rwth}%

\begin{frame}[fragile]
        \frametitle{Project Partners}
        \framesubtitle{How to set up partner logos}
        \begin{itemize}
      \item Show up to 3 partner logos, on this slide Jara, RWTH, Bonn
      \item Design your logos with sufficiently large white borders
      \item {pdf\LaTeX} pictures file types: \verb+.pdf .png .jpg+
    \end{itemize}
        \begin{block}{Show logos}
        \verb+\setbeamertemplate{footer element1}[logo]{jara}+
                \verb+\setbeamertemplate{footer element2}[logo]{uni_bonn}+
        \verb+\setbeamertemplate{footer element3}[logo]{rwth}+
    \end{block}
        \begin{block}{Reset back to default settings}
        \verb+\setbeamertemplate{footer element1}[default]+
        \verb+\setbeamertemplate{footer element2}[default]+
        \verb+\setbeamertemplate{footer element3}[default]+
    \end{block}
\end{frame}
\setbeamertemplate{footer element1}[default]
\setbeamertemplate{footer element2}[default]
\setbeamertemplate{footer element3}[default]


\section{Institute Logo}

{\setbeamertemplate{footer element4}[logo]{fzj-jsc}
\begin{frame}[fragile]
    \frametitle{FZJ Logo with Institute Name}
    \begin{itemize}
        \item A variant of the Jülich logo has the institute's name right next to it
        \item How this looks like for Jülich Supercomputing Centre is shown on this slide
        \item Changing the logo works through the mechanism presented on the previous slide
        \begin{block}{Change Jülich Logo}
            \verb+\setbeamertemplate{footer element4}[logo]{fzj-jsc}+
        \end{block}
        \item In contrast to the image insertion mechanism for \verb+footer element1-3+, the logo is vertically adjusted to the bottom baseline of the slide.
        \item The included logo, \verb+fzj-jsc+ in the example here, is expected to be a graphic without any surrounding whitespace
    \end{itemize}
\end{frame}
}%footer element JSC

\section{Progress Bar}

{%
\setbeamertemplate{progressbar}[visible]%
\fzjset{progress bar/height=0.4ex}%
\begin{frame}[fragile]
    \frametitle{Progress Bar}
    \begin{itemize}
        \item (Experimental) Support for progress bar
        \item Highly configurable, see \verb+progressbar.tex+
        \item Feedback welcome!
    \end{itemize}
\end{frame}
}
